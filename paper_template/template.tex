\documentclass{article}

%%%%%%%%%%%%%%%%%%%%%%%%%%%%%%%%%%%%%%%%%%%%%%%%%%%%%%%%%

%no packages
%no newly defined commands

%%%%%%%%%%%%%%%%%%%%%%%%%%%%%%%%%%%%%%%%%%%%%%%%%%%%%%%%%

\begin{document}

\title{my manuscript}

\date{}

\author{me\thanks{corresponding author} \thanks{my address}\and my best friend and co-author\thanks{address of best friend and co-author}\and the lazy guy who didn't do anything at all\footnotemark[3]\and my boss, because that is how one should do it\footnotemark[1] \footnotemark[2]}

\maketitle

\section{Abstract}

My manuscript is about interesting research, and this is what we found.

\section{Introduction}

This and that is known already \cite{lita, litb, litc, litd}. Herein, we present some other new results.

\section{Results and Discussion}

Measuring this we got that. See Figure \ref{fig1} and Table \ref{tab1}. With equation \ref{eq1} we calculate {\bf{x}} and put it into equation \ref{eq2}.

\begin{equation}
{\bf{x}}=a+\Delta G_{free}
\label{eq1}
\end{equation}

%%%%%%%%%%%%%%%%%%%%%%%%%%%%%%%%%%%%%%%%%%%%%%%%%%%%%%%%%

%if your equations are more complicated, such as:

\begin{equation}
\left[\begin{array}{c}
\psi^{L}\\
\psi^{S}
\end{array}\right]=\left[\begin{array}{cc}
I_{2} & 0_{2}\\
0_{2} & \frac{1}{2mc}\left(\sigma\cdot\bf{x}\right)
\end{array}\right]\left[\begin{array}{c}
\psi^{L}\\
\phi^{L}
\end{array}\right]
\label{eq2}
\end{equation}

%then send every equation in a separate (according to the equation number numbered).pdf file and we treat them as graphic files, set by the %typesetter.

%%%%%%%%%%%%%%%%%%%%%%%%%%%%%%%%%%%%%%%%%%%%%%%%%%%%%%%%%

\section{Conclusions}

Our results show that we were right and our ideas can be applied here and there.

\section{Experimental Section}

We did our experiments under these conditions using machines w and x, and chemicals y and z. For calculations we used the calculates-whatever-you-want program \cite{lite}.

\section{Acknowledgment}

We thank our research fund for the money.


\section{Keywords}

Keyword 1, keyword 2, keyword 3, keyword 4, keyword 5

\section{TOC}

This and that is shown (see picture \ref{TOCfig}) and can be used here and there.

\begin{thebibliography}{99}

\bibitem{lita} M. Mouse, D. Duck, {\em ChemPhysChem} {\bf 2005}, {\em 38}, 1764.
\bibitem{litb} S. Cooper, L. Hofstadter, {\em Whatever you wanted to know.}, (Eds.:C. Lorre, B. Prady), Wiley-VCH, Weinheim, {\bf 2003}, pp. 1658-2014.
\bibitem{litc} C. Brown (Peanuts Co.), patent number: US-A 549623, {\bf 2010}.
\bibitem{litd} C. Kent, {\em Chem. Eur. J.} {\bf 2012}, unpublished results.
\bibitem{lite} S. Brain, {\bf 2008}, http://www.calculates-whatever-you-want.com/maybe.

\end{thebibliography}

%%%%%%%%%%%%%%%%%%%%%%%%%%%%%%%%%%%%%%%%%%%%%%%%%%%%%%%%%

%alternatively you may use

%\bibliographystyle{unsrt}
%\bibliography{mymanuscript.bib}

%and send the .bib file along with your manuscript

%%%%%%%%%%%%%%%%%%%%%%%%%%%%%%%%%%%%%%%%%%%%%%%%%%%%%%%%%

\newpage

\begin{figure}
%\includegraphics{1}
\caption{This figure shows this and that.}
\label{fig1}
\end{figure}

\begin{figure}
%\includegraphics{TOC}
\caption{TOC figure.}
\label{TOCfig}
\end{figure}

\begin{table}
\begin{center}
\begin{tabular}{l l l l l l}
\hline
\multicolumn{6}{c}{My data collection.}\\
\hline
entry & bla & blabla & blabli & blablu & blah\\
\hline
this & is & the & first$^{[a]}$ & row & data \\
this & is & the & second & row & data \\
\hline
\end{tabular}
\end{center}
\caption{In this table my data is listed. [a] There is no zeroth row. }
\label{tab1}
\end{table}

\end{document}


